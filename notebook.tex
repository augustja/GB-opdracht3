
% Default to the notebook output style

    


% Inherit from the specified cell style.




    
\documentclass[11pt]{article}

    
    
    \usepackage[T1]{fontenc}
    % Nicer default font (+ math font) than Computer Modern for most use cases
    \usepackage{mathpazo}

    % Basic figure setup, for now with no caption control since it's done
    % automatically by Pandoc (which extracts ![](path) syntax from Markdown).
    \usepackage{graphicx}
    % We will generate all images so they have a width \maxwidth. This means
    % that they will get their normal width if they fit onto the page, but
    % are scaled down if they would overflow the margins.
    \makeatletter
    \def\maxwidth{\ifdim\Gin@nat@width>\linewidth\linewidth
    \else\Gin@nat@width\fi}
    \makeatother
    \let\Oldincludegraphics\includegraphics
    % Set max figure width to be 80% of text width, for now hardcoded.
    \renewcommand{\includegraphics}[1]{\Oldincludegraphics[width=.8\maxwidth]{#1}}
    % Ensure that by default, figures have no caption (until we provide a
    % proper Figure object with a Caption API and a way to capture that
    % in the conversion process - todo).
    \usepackage{caption}
    \DeclareCaptionLabelFormat{nolabel}{}
    \captionsetup{labelformat=nolabel}

    \usepackage{adjustbox} % Used to constrain images to a maximum size 
    \usepackage{xcolor} % Allow colors to be defined
    \usepackage{enumerate} % Needed for markdown enumerations to work
    \usepackage{geometry} % Used to adjust the document margins
    \usepackage{amsmath} % Equations
    \usepackage{amssymb} % Equations
    \usepackage{textcomp} % defines textquotesingle
    % Hack from http://tex.stackexchange.com/a/47451/13684:
    \AtBeginDocument{%
        \def\PYZsq{\textquotesingle}% Upright quotes in Pygmentized code
    }
    \usepackage{upquote} % Upright quotes for verbatim code
    \usepackage{eurosym} % defines \euro
    \usepackage[mathletters]{ucs} % Extended unicode (utf-8) support
    \usepackage[utf8x]{inputenc} % Allow utf-8 characters in the tex document
    \usepackage{fancyvrb} % verbatim replacement that allows latex
    \usepackage{grffile} % extends the file name processing of package graphics 
                         % to support a larger range 
    % The hyperref package gives us a pdf with properly built
    % internal navigation ('pdf bookmarks' for the table of contents,
    % internal cross-reference links, web links for URLs, etc.)
    \usepackage{hyperref}
    \usepackage{longtable} % longtable support required by pandoc >1.10
    \usepackage{booktabs}  % table support for pandoc > 1.12.2
    \usepackage[inline]{enumitem} % IRkernel/repr support (it uses the enumerate* environment)
    \usepackage[normalem]{ulem} % ulem is needed to support strikethroughs (\sout)
                                % normalem makes italics be italics, not underlines
    

    
    
    % Colors for the hyperref package
    \definecolor{urlcolor}{rgb}{0,.145,.698}
    \definecolor{linkcolor}{rgb}{.71,0.21,0.01}
    \definecolor{citecolor}{rgb}{.12,.54,.11}

    % ANSI colors
    \definecolor{ansi-black}{HTML}{3E424D}
    \definecolor{ansi-black-intense}{HTML}{282C36}
    \definecolor{ansi-red}{HTML}{E75C58}
    \definecolor{ansi-red-intense}{HTML}{B22B31}
    \definecolor{ansi-green}{HTML}{00A250}
    \definecolor{ansi-green-intense}{HTML}{007427}
    \definecolor{ansi-yellow}{HTML}{DDB62B}
    \definecolor{ansi-yellow-intense}{HTML}{B27D12}
    \definecolor{ansi-blue}{HTML}{208FFB}
    \definecolor{ansi-blue-intense}{HTML}{0065CA}
    \definecolor{ansi-magenta}{HTML}{D160C4}
    \definecolor{ansi-magenta-intense}{HTML}{A03196}
    \definecolor{ansi-cyan}{HTML}{60C6C8}
    \definecolor{ansi-cyan-intense}{HTML}{258F8F}
    \definecolor{ansi-white}{HTML}{C5C1B4}
    \definecolor{ansi-white-intense}{HTML}{A1A6B2}

    % commands and environments needed by pandoc snippets
    % extracted from the output of `pandoc -s`
    \providecommand{\tightlist}{%
      \setlength{\itemsep}{0pt}\setlength{\parskip}{0pt}}
    \DefineVerbatimEnvironment{Highlighting}{Verbatim}{commandchars=\\\{\}}
    % Add ',fontsize=\small' for more characters per line
    \newenvironment{Shaded}{}{}
    \newcommand{\KeywordTok}[1]{\textcolor[rgb]{0.00,0.44,0.13}{\textbf{{#1}}}}
    \newcommand{\DataTypeTok}[1]{\textcolor[rgb]{0.56,0.13,0.00}{{#1}}}
    \newcommand{\DecValTok}[1]{\textcolor[rgb]{0.25,0.63,0.44}{{#1}}}
    \newcommand{\BaseNTok}[1]{\textcolor[rgb]{0.25,0.63,0.44}{{#1}}}
    \newcommand{\FloatTok}[1]{\textcolor[rgb]{0.25,0.63,0.44}{{#1}}}
    \newcommand{\CharTok}[1]{\textcolor[rgb]{0.25,0.44,0.63}{{#1}}}
    \newcommand{\StringTok}[1]{\textcolor[rgb]{0.25,0.44,0.63}{{#1}}}
    \newcommand{\CommentTok}[1]{\textcolor[rgb]{0.38,0.63,0.69}{\textit{{#1}}}}
    \newcommand{\OtherTok}[1]{\textcolor[rgb]{0.00,0.44,0.13}{{#1}}}
    \newcommand{\AlertTok}[1]{\textcolor[rgb]{1.00,0.00,0.00}{\textbf{{#1}}}}
    \newcommand{\FunctionTok}[1]{\textcolor[rgb]{0.02,0.16,0.49}{{#1}}}
    \newcommand{\RegionMarkerTok}[1]{{#1}}
    \newcommand{\ErrorTok}[1]{\textcolor[rgb]{1.00,0.00,0.00}{\textbf{{#1}}}}
    \newcommand{\NormalTok}[1]{{#1}}
    
    % Additional commands for more recent versions of Pandoc
    \newcommand{\ConstantTok}[1]{\textcolor[rgb]{0.53,0.00,0.00}{{#1}}}
    \newcommand{\SpecialCharTok}[1]{\textcolor[rgb]{0.25,0.44,0.63}{{#1}}}
    \newcommand{\VerbatimStringTok}[1]{\textcolor[rgb]{0.25,0.44,0.63}{{#1}}}
    \newcommand{\SpecialStringTok}[1]{\textcolor[rgb]{0.73,0.40,0.53}{{#1}}}
    \newcommand{\ImportTok}[1]{{#1}}
    \newcommand{\DocumentationTok}[1]{\textcolor[rgb]{0.73,0.13,0.13}{\textit{{#1}}}}
    \newcommand{\AnnotationTok}[1]{\textcolor[rgb]{0.38,0.63,0.69}{\textbf{\textit{{#1}}}}}
    \newcommand{\CommentVarTok}[1]{\textcolor[rgb]{0.38,0.63,0.69}{\textbf{\textit{{#1}}}}}
    \newcommand{\VariableTok}[1]{\textcolor[rgb]{0.10,0.09,0.49}{{#1}}}
    \newcommand{\ControlFlowTok}[1]{\textcolor[rgb]{0.00,0.44,0.13}{\textbf{{#1}}}}
    \newcommand{\OperatorTok}[1]{\textcolor[rgb]{0.40,0.40,0.40}{{#1}}}
    \newcommand{\BuiltInTok}[1]{{#1}}
    \newcommand{\ExtensionTok}[1]{{#1}}
    \newcommand{\PreprocessorTok}[1]{\textcolor[rgb]{0.74,0.48,0.00}{{#1}}}
    \newcommand{\AttributeTok}[1]{\textcolor[rgb]{0.49,0.56,0.16}{{#1}}}
    \newcommand{\InformationTok}[1]{\textcolor[rgb]{0.38,0.63,0.69}{\textbf{\textit{{#1}}}}}
    \newcommand{\WarningTok}[1]{\textcolor[rgb]{0.38,0.63,0.69}{\textbf{\textit{{#1}}}}}
    
    
    % Define a nice break command that doesn't care if a line doesn't already
    % exist.
    \def\br{\hspace*{\fill} \\* }
    % Math Jax compatability definitions
    \def\gt{>}
    \def\lt{<}
    % Document parameters
    \title{04 - ASSIGNMENT(3)}
    
    
    

    % Pygments definitions
    
\makeatletter
\def\PY@reset{\let\PY@it=\relax \let\PY@bf=\relax%
    \let\PY@ul=\relax \let\PY@tc=\relax%
    \let\PY@bc=\relax \let\PY@ff=\relax}
\def\PY@tok#1{\csname PY@tok@#1\endcsname}
\def\PY@toks#1+{\ifx\relax#1\empty\else%
    \PY@tok{#1}\expandafter\PY@toks\fi}
\def\PY@do#1{\PY@bc{\PY@tc{\PY@ul{%
    \PY@it{\PY@bf{\PY@ff{#1}}}}}}}
\def\PY#1#2{\PY@reset\PY@toks#1+\relax+\PY@do{#2}}

\expandafter\def\csname PY@tok@w\endcsname{\def\PY@tc##1{\textcolor[rgb]{0.73,0.73,0.73}{##1}}}
\expandafter\def\csname PY@tok@c\endcsname{\let\PY@it=\textit\def\PY@tc##1{\textcolor[rgb]{0.25,0.50,0.50}{##1}}}
\expandafter\def\csname PY@tok@cp\endcsname{\def\PY@tc##1{\textcolor[rgb]{0.74,0.48,0.00}{##1}}}
\expandafter\def\csname PY@tok@k\endcsname{\let\PY@bf=\textbf\def\PY@tc##1{\textcolor[rgb]{0.00,0.50,0.00}{##1}}}
\expandafter\def\csname PY@tok@kp\endcsname{\def\PY@tc##1{\textcolor[rgb]{0.00,0.50,0.00}{##1}}}
\expandafter\def\csname PY@tok@kt\endcsname{\def\PY@tc##1{\textcolor[rgb]{0.69,0.00,0.25}{##1}}}
\expandafter\def\csname PY@tok@o\endcsname{\def\PY@tc##1{\textcolor[rgb]{0.40,0.40,0.40}{##1}}}
\expandafter\def\csname PY@tok@ow\endcsname{\let\PY@bf=\textbf\def\PY@tc##1{\textcolor[rgb]{0.67,0.13,1.00}{##1}}}
\expandafter\def\csname PY@tok@nb\endcsname{\def\PY@tc##1{\textcolor[rgb]{0.00,0.50,0.00}{##1}}}
\expandafter\def\csname PY@tok@nf\endcsname{\def\PY@tc##1{\textcolor[rgb]{0.00,0.00,1.00}{##1}}}
\expandafter\def\csname PY@tok@nc\endcsname{\let\PY@bf=\textbf\def\PY@tc##1{\textcolor[rgb]{0.00,0.00,1.00}{##1}}}
\expandafter\def\csname PY@tok@nn\endcsname{\let\PY@bf=\textbf\def\PY@tc##1{\textcolor[rgb]{0.00,0.00,1.00}{##1}}}
\expandafter\def\csname PY@tok@ne\endcsname{\let\PY@bf=\textbf\def\PY@tc##1{\textcolor[rgb]{0.82,0.25,0.23}{##1}}}
\expandafter\def\csname PY@tok@nv\endcsname{\def\PY@tc##1{\textcolor[rgb]{0.10,0.09,0.49}{##1}}}
\expandafter\def\csname PY@tok@no\endcsname{\def\PY@tc##1{\textcolor[rgb]{0.53,0.00,0.00}{##1}}}
\expandafter\def\csname PY@tok@nl\endcsname{\def\PY@tc##1{\textcolor[rgb]{0.63,0.63,0.00}{##1}}}
\expandafter\def\csname PY@tok@ni\endcsname{\let\PY@bf=\textbf\def\PY@tc##1{\textcolor[rgb]{0.60,0.60,0.60}{##1}}}
\expandafter\def\csname PY@tok@na\endcsname{\def\PY@tc##1{\textcolor[rgb]{0.49,0.56,0.16}{##1}}}
\expandafter\def\csname PY@tok@nt\endcsname{\let\PY@bf=\textbf\def\PY@tc##1{\textcolor[rgb]{0.00,0.50,0.00}{##1}}}
\expandafter\def\csname PY@tok@nd\endcsname{\def\PY@tc##1{\textcolor[rgb]{0.67,0.13,1.00}{##1}}}
\expandafter\def\csname PY@tok@s\endcsname{\def\PY@tc##1{\textcolor[rgb]{0.73,0.13,0.13}{##1}}}
\expandafter\def\csname PY@tok@sd\endcsname{\let\PY@it=\textit\def\PY@tc##1{\textcolor[rgb]{0.73,0.13,0.13}{##1}}}
\expandafter\def\csname PY@tok@si\endcsname{\let\PY@bf=\textbf\def\PY@tc##1{\textcolor[rgb]{0.73,0.40,0.53}{##1}}}
\expandafter\def\csname PY@tok@se\endcsname{\let\PY@bf=\textbf\def\PY@tc##1{\textcolor[rgb]{0.73,0.40,0.13}{##1}}}
\expandafter\def\csname PY@tok@sr\endcsname{\def\PY@tc##1{\textcolor[rgb]{0.73,0.40,0.53}{##1}}}
\expandafter\def\csname PY@tok@ss\endcsname{\def\PY@tc##1{\textcolor[rgb]{0.10,0.09,0.49}{##1}}}
\expandafter\def\csname PY@tok@sx\endcsname{\def\PY@tc##1{\textcolor[rgb]{0.00,0.50,0.00}{##1}}}
\expandafter\def\csname PY@tok@m\endcsname{\def\PY@tc##1{\textcolor[rgb]{0.40,0.40,0.40}{##1}}}
\expandafter\def\csname PY@tok@gh\endcsname{\let\PY@bf=\textbf\def\PY@tc##1{\textcolor[rgb]{0.00,0.00,0.50}{##1}}}
\expandafter\def\csname PY@tok@gu\endcsname{\let\PY@bf=\textbf\def\PY@tc##1{\textcolor[rgb]{0.50,0.00,0.50}{##1}}}
\expandafter\def\csname PY@tok@gd\endcsname{\def\PY@tc##1{\textcolor[rgb]{0.63,0.00,0.00}{##1}}}
\expandafter\def\csname PY@tok@gi\endcsname{\def\PY@tc##1{\textcolor[rgb]{0.00,0.63,0.00}{##1}}}
\expandafter\def\csname PY@tok@gr\endcsname{\def\PY@tc##1{\textcolor[rgb]{1.00,0.00,0.00}{##1}}}
\expandafter\def\csname PY@tok@ge\endcsname{\let\PY@it=\textit}
\expandafter\def\csname PY@tok@gs\endcsname{\let\PY@bf=\textbf}
\expandafter\def\csname PY@tok@gp\endcsname{\let\PY@bf=\textbf\def\PY@tc##1{\textcolor[rgb]{0.00,0.00,0.50}{##1}}}
\expandafter\def\csname PY@tok@go\endcsname{\def\PY@tc##1{\textcolor[rgb]{0.53,0.53,0.53}{##1}}}
\expandafter\def\csname PY@tok@gt\endcsname{\def\PY@tc##1{\textcolor[rgb]{0.00,0.27,0.87}{##1}}}
\expandafter\def\csname PY@tok@err\endcsname{\def\PY@bc##1{\setlength{\fboxsep}{0pt}\fcolorbox[rgb]{1.00,0.00,0.00}{1,1,1}{\strut ##1}}}
\expandafter\def\csname PY@tok@kc\endcsname{\let\PY@bf=\textbf\def\PY@tc##1{\textcolor[rgb]{0.00,0.50,0.00}{##1}}}
\expandafter\def\csname PY@tok@kd\endcsname{\let\PY@bf=\textbf\def\PY@tc##1{\textcolor[rgb]{0.00,0.50,0.00}{##1}}}
\expandafter\def\csname PY@tok@kn\endcsname{\let\PY@bf=\textbf\def\PY@tc##1{\textcolor[rgb]{0.00,0.50,0.00}{##1}}}
\expandafter\def\csname PY@tok@kr\endcsname{\let\PY@bf=\textbf\def\PY@tc##1{\textcolor[rgb]{0.00,0.50,0.00}{##1}}}
\expandafter\def\csname PY@tok@bp\endcsname{\def\PY@tc##1{\textcolor[rgb]{0.00,0.50,0.00}{##1}}}
\expandafter\def\csname PY@tok@fm\endcsname{\def\PY@tc##1{\textcolor[rgb]{0.00,0.00,1.00}{##1}}}
\expandafter\def\csname PY@tok@vc\endcsname{\def\PY@tc##1{\textcolor[rgb]{0.10,0.09,0.49}{##1}}}
\expandafter\def\csname PY@tok@vg\endcsname{\def\PY@tc##1{\textcolor[rgb]{0.10,0.09,0.49}{##1}}}
\expandafter\def\csname PY@tok@vi\endcsname{\def\PY@tc##1{\textcolor[rgb]{0.10,0.09,0.49}{##1}}}
\expandafter\def\csname PY@tok@vm\endcsname{\def\PY@tc##1{\textcolor[rgb]{0.10,0.09,0.49}{##1}}}
\expandafter\def\csname PY@tok@sa\endcsname{\def\PY@tc##1{\textcolor[rgb]{0.73,0.13,0.13}{##1}}}
\expandafter\def\csname PY@tok@sb\endcsname{\def\PY@tc##1{\textcolor[rgb]{0.73,0.13,0.13}{##1}}}
\expandafter\def\csname PY@tok@sc\endcsname{\def\PY@tc##1{\textcolor[rgb]{0.73,0.13,0.13}{##1}}}
\expandafter\def\csname PY@tok@dl\endcsname{\def\PY@tc##1{\textcolor[rgb]{0.73,0.13,0.13}{##1}}}
\expandafter\def\csname PY@tok@s2\endcsname{\def\PY@tc##1{\textcolor[rgb]{0.73,0.13,0.13}{##1}}}
\expandafter\def\csname PY@tok@sh\endcsname{\def\PY@tc##1{\textcolor[rgb]{0.73,0.13,0.13}{##1}}}
\expandafter\def\csname PY@tok@s1\endcsname{\def\PY@tc##1{\textcolor[rgb]{0.73,0.13,0.13}{##1}}}
\expandafter\def\csname PY@tok@mb\endcsname{\def\PY@tc##1{\textcolor[rgb]{0.40,0.40,0.40}{##1}}}
\expandafter\def\csname PY@tok@mf\endcsname{\def\PY@tc##1{\textcolor[rgb]{0.40,0.40,0.40}{##1}}}
\expandafter\def\csname PY@tok@mh\endcsname{\def\PY@tc##1{\textcolor[rgb]{0.40,0.40,0.40}{##1}}}
\expandafter\def\csname PY@tok@mi\endcsname{\def\PY@tc##1{\textcolor[rgb]{0.40,0.40,0.40}{##1}}}
\expandafter\def\csname PY@tok@il\endcsname{\def\PY@tc##1{\textcolor[rgb]{0.40,0.40,0.40}{##1}}}
\expandafter\def\csname PY@tok@mo\endcsname{\def\PY@tc##1{\textcolor[rgb]{0.40,0.40,0.40}{##1}}}
\expandafter\def\csname PY@tok@ch\endcsname{\let\PY@it=\textit\def\PY@tc##1{\textcolor[rgb]{0.25,0.50,0.50}{##1}}}
\expandafter\def\csname PY@tok@cm\endcsname{\let\PY@it=\textit\def\PY@tc##1{\textcolor[rgb]{0.25,0.50,0.50}{##1}}}
\expandafter\def\csname PY@tok@cpf\endcsname{\let\PY@it=\textit\def\PY@tc##1{\textcolor[rgb]{0.25,0.50,0.50}{##1}}}
\expandafter\def\csname PY@tok@c1\endcsname{\let\PY@it=\textit\def\PY@tc##1{\textcolor[rgb]{0.25,0.50,0.50}{##1}}}
\expandafter\def\csname PY@tok@cs\endcsname{\let\PY@it=\textit\def\PY@tc##1{\textcolor[rgb]{0.25,0.50,0.50}{##1}}}

\def\PYZbs{\char`\\}
\def\PYZus{\char`\_}
\def\PYZob{\char`\{}
\def\PYZcb{\char`\}}
\def\PYZca{\char`\^}
\def\PYZam{\char`\&}
\def\PYZlt{\char`\<}
\def\PYZgt{\char`\>}
\def\PYZsh{\char`\#}
\def\PYZpc{\char`\%}
\def\PYZdl{\char`\$}
\def\PYZhy{\char`\-}
\def\PYZsq{\char`\'}
\def\PYZdq{\char`\"}
\def\PYZti{\char`\~}
% for compatibility with earlier versions
\def\PYZat{@}
\def\PYZlb{[}
\def\PYZrb{]}
\makeatother


    % Exact colors from NB
    \definecolor{incolor}{rgb}{0.0, 0.0, 0.5}
    \definecolor{outcolor}{rgb}{0.545, 0.0, 0.0}



    
    % Prevent overflowing lines due to hard-to-break entities
    \sloppy 
    % Setup hyperref package
    \hypersetup{
      breaklinks=true,  % so long urls are correctly broken across lines
      colorlinks=true,
      urlcolor=urlcolor,
      linkcolor=linkcolor,
      citecolor=citecolor,
      }
    % Slightly bigger margins than the latex defaults
    
    \geometry{verbose,tmargin=1in,bmargin=1in,lmargin=1in,rmargin=1in}
    
    

    \begin{document}
    
    
    \maketitle
    
    

    
    \section{Werkje gegevensbanken 2018 - DEEL
04}\label{werkje-gegevensbanken-2018---deel-04}

\subsection{Inleiding}\label{inleiding}

Het vierde en laatste deel van het werkje bestaat uit twee componenten:
1. Optimalisatie 2. Visualisatie

Bij dit deel van het werkje laten we jullie veel vrijheid om tot een
oplossing te komen. Er bestaat niet zoiets als de unieke correcte
oplossing, het is dus de bedoeling dat jullie zelf een strategie
uitdokteren en deze achteraf kunnen verdedigen. Leg in deze notebook
kort uit welke keuzes je maakt en waarom, dit zal helpen bij de
verdediging. Beschouw dit als een soort mini-verslag tussen de code
door.

\subsubsection{Indienen +
Evaluatiemoment}\label{indienen-evaluatiemoment}

** EVALULATIEMOMENT**: DE OEFENZITTING IN DE WEEK VAN
14/05/2017-18/05/2017

Het evaluatiemoment van het \textbf{volledige} werkje vindt plaats
tijdens \textbf{de oefenzitting in de week van 14/05/2017-18/05/2017}.
Prof. Berendt en dr. Bogaerts komen langs om enkele vragen te stellen
over jullie oplossingen. Dit evaluatiemoment omvat ook een demonstratie
van jullie resultaat, bijvoorbeeld: aantonen dat jullie optimalisatie
van deel 4 van het werkje effectief een verbetering is t.o.v. het
originele geval. Teneinde die demonstratie vlot te doen verlopen vragen
we jullie om \textbf{per groep (minstens) 1 laptop mee te nemen waarop
jullie oplossingen (i.e. de ingevulde notebooks) goed functioneren}.
Deze laptop beschikt dus ook over XAMPP (inclusief de gegevensbank van
dit werkje, uiteraard), gezien de queries in de notebooks ook moeten
werken.

\textbf{INDIENDEADLINE}: ZONDAG 13 MEI om 23.59

** HOE INDIENEN**: 1. Upload je ingevulde notebook als
werkje\_04\_groep\_XX.ipynb EN werkje\_04\_groep\_XX.html 2. Vervang XX
door je groepsnummer. 3. Uploaden doe je op Toledo in je groepsfolder.

Zorg ervoor dat de .html file zeker ook de output van de visualisatie
bevat, zodat wij deze gemakkelijk kunnen bekijken en niet moeten
reproduceren door jullie .ipynb te runnen! Op die manier kunnen we zeker
zijn dat we evalueren wat jullie geproduceerd hebben.

    \subsubsection{Nuttige packages en functies
inladen}\label{nuttige-packages-en-functies-inladen}

Enkele nuttige packages en functies inladen die verderop van pas zullen
komen.

    \begin{Verbatim}[commandchars=\\\{\}]
{\color{incolor}In [{\color{incolor}1}]:} \PY{c+c1}{\PYZsh{} Benodigde packages}
        \PY{k+kn}{import} \PY{n+nn}{json}            \PY{c+c1}{\PYZsh{} Package om .json files in te laden (bvb kolomnamen zijn zo opgeslagen)}
        \PY{k+kn}{import} \PY{n+nn}{getpass}         \PY{c+c1}{\PYZsh{} Package om een paswoordveldje te genereren.}
        \PY{k+kn}{import} \PY{n+nn}{mysql}\PY{n+nn}{.}\PY{n+nn}{connector} \PY{c+c1}{\PYZsh{} MySQL package}
        \PY{k+kn}{import} \PY{n+nn}{numpy} \PY{k}{as} \PY{n+nn}{np}
        \PY{k+kn}{import} \PY{n+nn}{pandas} \PY{k}{as} \PY{n+nn}{pd}    \PY{c+c1}{\PYZsh{} Populaire package voor data\PYZhy{}verwerking}
        \PY{k+kn}{import} \PY{n+nn}{sys}
        \PY{k+kn}{import} \PY{n+nn}{os}
        \PY{k+kn}{import} \PY{n+nn}{time}
        \PY{c+c1}{\PYZsh{} Benodigde packages voor visualisatie in Seaborn}
        \PY{k+kn}{import} \PY{n+nn}{matplotlib}\PY{n+nn}{.}\PY{n+nn}{pyplot} \PY{k}{as} \PY{n+nn}{plt}
        \PY{k+kn}{import} \PY{n+nn}{seaborn} \PY{k}{as} \PY{n+nn}{sns}
\end{Verbatim}


    \begin{Verbatim}[commandchars=\\\{\}]
{\color{incolor}In [{\color{incolor}2}]:} \PY{n}{sys}\PY{o}{.}\PY{n}{version\PYZus{}info}       \PY{c+c1}{\PYZsh{} Check python versie, wij veronderstellen 3.6}
\end{Verbatim}


\begin{Verbatim}[commandchars=\\\{\}]
{\color{outcolor}Out[{\color{outcolor}2}]:} sys.version\_info(major=3, minor=6, micro=4, releaselevel='final', serial=0)
\end{Verbatim}
            
    \subsubsection{Interageren met een
gegevensbank}\label{interageren-met-een-gegevensbank}

    \begin{Verbatim}[commandchars=\\\{\}]
{\color{incolor}In [{\color{incolor}3}]:} \PY{k}{def} \PY{n+nf}{verbind\PYZus{}met\PYZus{}GB}\PY{p}{(}\PY{n}{username}\PY{p}{,} \PY{n}{hostname}\PY{p}{,} \PY{n}{gegevensbanknaam}\PY{p}{)}\PY{p}{:}
            \PY{l+s+sd}{\PYZdq{}\PYZdq{}\PYZdq{}}
        \PY{l+s+sd}{    Maak verbinding met een externe gegevensbank}
        \PY{l+s+sd}{    }
        \PY{l+s+sd}{    :param  username:          username van de gebruiker, string}
        \PY{l+s+sd}{    :param  hostname:          naam van de host, string.}
        \PY{l+s+sd}{                               Dit is in het geval van een lokale server gewoon \PYZsq{}localhost\PYZsq{}}
        \PY{l+s+sd}{    :param  gegevensbanknaam:  naam van de gegevensbank, string.}
        \PY{l+s+sd}{    :return connection:        connection object, dit is wat teruggeven wordt }
        \PY{l+s+sd}{                               door connect() methods van packages die voldoen aan de DB\PYZhy{}API}
        \PY{l+s+sd}{    \PYZdq{}\PYZdq{}\PYZdq{}}
            
            \PY{n}{password} \PY{o}{=} \PY{n}{getpass}\PY{o}{.}\PY{n}{getpass}\PY{p}{(}\PY{p}{)} \PY{c+c1}{\PYZsh{} Genereer vakje voor wachtwoord in te geven}
            
            \PY{n}{connection} \PY{o}{=} \PY{n}{mysql}\PY{o}{.}\PY{n}{connector}\PY{o}{.}\PY{n}{connect}\PY{p}{(}\PY{n}{host}\PY{o}{=}\PY{n}{hostname}\PY{p}{,}
                                                 \PY{n}{user}\PY{o}{=}\PY{n}{username}\PY{p}{,}
                                                 \PY{n}{passwd}\PY{o}{=}\PY{n}{password}\PY{p}{,}
                                                 \PY{n}{db}\PY{o}{=}\PY{n}{gegevensbanknaam}\PY{p}{)}
            \PY{k}{return} \PY{n}{connection}
        
        
        \PY{k}{def} \PY{n+nf}{check\PYZus{}perfect\PYZus{}match}\PY{p}{(}\PY{n}{df1}\PY{p}{,} \PY{n}{df2}\PY{p}{)}\PY{p}{:}
            \PY{l+s+sd}{\PYZdq{}\PYZdq{}\PYZdq{}}
        \PY{l+s+sd}{    Functie om te checken of 2 DataFrames gelijk zijn.}
        \PY{l+s+sd}{    \PYZdq{}\PYZdq{}\PYZdq{}}
            \PY{n}{check} \PY{o}{=} \PY{n}{df1}\PY{o}{.}\PY{n}{equals}\PY{p}{(}\PY{n}{df2}\PY{p}{)}
            \PY{k}{return} \PY{n}{check}
        
        
        \PY{k}{def} \PY{n+nf}{run\PYZus{}query}\PY{p}{(}\PY{n}{connection}\PY{p}{,} \PY{n}{query}\PY{p}{)}\PY{p}{:}
            \PY{l+s+sd}{\PYZdq{}\PYZdq{}\PYZdq{}}
        \PY{l+s+sd}{    Voer een query uit op een reeds gemaakte connectie, geeft het resultaat van de query terug}
        \PY{l+s+sd}{    \PYZdq{}\PYZdq{}\PYZdq{}}
            
            \PY{c+c1}{\PYZsh{} Making a cursor and executing the query}
            \PY{n}{cursor} \PY{o}{=} \PY{n}{connection}\PY{o}{.}\PY{n}{cursor}\PY{p}{(}\PY{p}{)}
            \PY{n}{cursor}\PY{o}{.}\PY{n}{execute}\PY{p}{(}\PY{n}{query}\PY{p}{)}
            
            \PY{c+c1}{\PYZsh{} Collecting the result and casting it in a pd.DataFrame}
            \PY{n}{res} \PY{o}{=} \PY{n}{cursor}\PY{o}{.}\PY{n}{fetchall}\PY{p}{(}\PY{p}{)}
            
            \PY{k}{return} \PY{n}{res}
        
        
        \PY{k}{def} \PY{n+nf}{res\PYZus{}to\PYZus{}df}\PY{p}{(}\PY{n}{query\PYZus{}result}\PY{p}{,} \PY{n}{column\PYZus{}names}\PY{p}{)}\PY{p}{:}
            \PY{l+s+sd}{\PYZdq{}\PYZdq{}\PYZdq{}}
        \PY{l+s+sd}{    Giet het resultaat van een uitgevoerde query in een \PYZsq{}pandas dataframe\PYZsq{}}
        \PY{l+s+sd}{    met vooraf gespecifieerde kolomnamen.}
        \PY{l+s+sd}{    }
        \PY{l+s+sd}{    Let op: Het resultaat van de query moet dus exact evenveel kolommen bevatten}
        \PY{l+s+sd}{    als kolomnamen die je meegeeft. Als dit niet het geval is, is dit een indicatie}
        \PY{l+s+sd}{    dat je oplossing fout is. (Gezien wij de kolomnamen van de oplossing al cadeau doen)}
        \PY{l+s+sd}{    }
        \PY{l+s+sd}{    \PYZdq{}\PYZdq{}\PYZdq{}}
            \PY{n}{df} \PY{o}{=} \PY{n}{pd}\PY{o}{.}\PY{n}{DataFrame}\PY{p}{(}\PY{n}{query\PYZus{}result}\PY{p}{,} \PY{n}{columns}\PY{o}{=}\PY{n}{column\PYZus{}names}\PY{p}{)}
            \PY{k}{return} \PY{n}{df}
\end{Verbatim}


    \subsection{Optimalisatie}\label{optimalisatie}

    \subsubsection{Opgave}\label{opgave}

Het doel van het eerste deel van deze taak is het optimaliseren van
onderstaande query.

Een goede oplossing van deze eerste opgave voldoet aan volgende
criteria: * Verzin een optimalisatie van deze query * De runtime van de
optimalisatie is significant minder dan die van de originele oplossing
(een paar procent is te weinig) * Je bent in staat om uit te leggen WAT
je doet en WAAROM je dat gedaan hebt en WAAROM het werkt.. * Het EXPLAIN
statement in MySQL is een goede gids om je hierbij te helpen. * Je
schrijft in deze notebook al kort de redeneringen op die achter jullie
optimalisatie zitten (dit zal nuttig zijn bij de mondelinge
verdediging).

Hieronder vinden jullie de query die te optimaliseren is. Wijzig deze
dus niet.

    \textbf{Beschrijving}

    Het resultaat van deze functie is een Pandas DataFrame dat voor een
gegeven \emph{jaar\_1} een aantal statistieken van alle staten bevat
waarbij de gemiddelde lengte van alle spelers geboren in die staat en
opgenomen in de hall of fame na \emph{jaar\_2} groter is dan
\emph{lengte}.

Voor die staten moet de tabel de volgende statistieken bevatten: het
gemiddelde gewicht, de gemiddelde lengte, het gemiddeld aantal batting
homeruns, en het gemiddeld aantal pitching saves van alle spelers
(geboren in die staat) die in de hall of fame zijn opgenomen na
\emph{jaar\_2}.

Sorteer oplopend alfabetisch op staat.

Nb. Lengte is uitgedrukt in inches.

    \begin{Verbatim}[commandchars=\\\{\}]
{\color{incolor}In [{\color{incolor}4}]:} \PY{n}{column\PYZus{}names} \PY{o}{=} \PY{p}{[}\PY{l+s+s1}{\PYZsq{}}\PY{l+s+s1}{state}\PY{l+s+s1}{\PYZsq{}}\PY{p}{,} \PY{l+s+s1}{\PYZsq{}}\PY{l+s+s1}{avg\PYZus{}weight}\PY{l+s+s1}{\PYZsq{}}\PY{p}{,} \PY{l+s+s1}{\PYZsq{}}\PY{l+s+s1}{avg\PYZus{}height}\PY{l+s+s1}{\PYZsq{}}\PY{p}{,} \PY{l+s+s1}{\PYZsq{}}\PY{l+s+s1}{avg\PYZus{}homeruns}\PY{l+s+s1}{\PYZsq{}}\PY{p}{,} \PY{l+s+s1}{\PYZsq{}}\PY{l+s+s1}{avg\PYZus{}saves}\PY{l+s+s1}{\PYZsq{}}\PY{p}{]}
\end{Verbatim}


    \begin{center}\rule{0.5\linewidth}{\linethickness}\end{center}

De query boom bepaald door de originele query zonder enige
optmalisaties.

    

    \begin{Verbatim}[commandchars=\\\{\}]
{\color{incolor}In [{\color{incolor}5}]:} \PY{c+c1}{\PYZsh{} Dit is de originele query die geoptimaliseerd dient te worden. Niet wijzigen!}
        \PY{k}{def} \PY{n+nf}{query\PYZus{}to\PYZus{}optimize}\PY{p}{(}\PY{n}{connection}\PY{p}{,} \PY{n}{column\PYZus{}names}\PY{p}{,} \PY{n}{jaar\PYZus{}1}\PY{o}{=}\PY{l+m+mi}{2000}\PY{p}{,} \PY{n}{jaar\PYZus{}2}\PY{o}{=}\PY{l+m+mi}{1990}\PY{p}{,} \PY{n}{lengte}\PY{o}{=}\PY{l+m+mi}{75}\PY{p}{)}\PY{p}{:}
            \PY{c+c1}{\PYZsh{} Actual query}
            \PY{n}{query}\PY{o}{=}\PY{l+s+s2}{\PYZdq{}\PYZdq{}\PYZdq{}}
        \PY{l+s+s2}{    SELECT m.birthState, AVG(m.weight), AVG(m.height), AVG(bat.HR), AVG(pit.SV)}
        \PY{l+s+s2}{    FROM Master AS m,}
        \PY{l+s+s2}{        Pitching AS pit,}
        \PY{l+s+s2}{        Batting AS bat,}
        \PY{l+s+s2}{        HallOfFame AS hof}
        \PY{l+s+s2}{    WHERE pit.yearID = }\PY{l+s+si}{\PYZob{}\PYZcb{}}
        \PY{l+s+s2}{        AND bat.yearID = }\PY{l+s+si}{\PYZob{}\PYZcb{}}
        \PY{l+s+s2}{        AND pit.playerID = m.playerID}
        \PY{l+s+s2}{        AND bat.playerID = m.playerID}
        \PY{l+s+s2}{        AND m.playerID = hof.playerID}
        \PY{l+s+s2}{        AND hof.yearID \PYZgt{} }\PY{l+s+si}{\PYZob{}\PYZcb{}}
        \PY{l+s+s2}{    GROUP BY m.birthState}
        \PY{l+s+s2}{    HAVING AVG(m.height) \PYZgt{} }\PY{l+s+si}{\PYZob{}\PYZcb{}}
        \PY{l+s+s2}{    ORDER BY m.birthState ASC;}
        \PY{l+s+s2}{    }\PY{l+s+s2}{\PYZdq{}\PYZdq{}\PYZdq{}}\PY{o}{.}\PY{n}{format}\PY{p}{(}\PY{n}{jaar\PYZus{}1}\PY{p}{,} \PY{n}{jaar\PYZus{}1}\PY{p}{,} \PY{n}{jaar\PYZus{}2}\PY{p}{,} \PY{n}{lengte}\PY{p}{)}
            
            \PY{c+c1}{\PYZsh{} Stap 2 \PYZam{} 3}
            \PY{n}{res} \PY{o}{=} \PY{n}{run\PYZus{}query}\PY{p}{(}\PY{n}{connection}\PY{p}{,} \PY{n}{query}\PY{p}{)}         \PY{c+c1}{\PYZsh{} Query uitvoeren}
            \PY{n}{df} \PY{o}{=} \PY{n}{res\PYZus{}to\PYZus{}df}\PY{p}{(}\PY{n}{res}\PY{p}{,} \PY{n}{column\PYZus{}names}\PY{p}{)}          \PY{c+c1}{\PYZsh{} Query in DataFrame brengen}
            
            \PY{k}{return} \PY{n}{df}
\end{Verbatim}


    Eerst maken we verbinding met de gegevensbank, zoals jullie al deden in
het vorige deel van het werkje.

    \begin{Verbatim}[commandchars=\\\{\}]
{\color{incolor}In [{\color{incolor}6}]:} \PY{n}{username} \PY{o}{=} \PY{l+s+s1}{\PYZsq{}}\PY{l+s+s1}{root}\PY{l+s+s1}{\PYZsq{}}      \PY{c+c1}{\PYZsh{} Vervang dit als je via een andere user queries stuurt}
        \PY{n}{hostname} \PY{o}{=} \PY{l+s+s1}{\PYZsq{}}\PY{l+s+s1}{localhost}\PY{l+s+s1}{\PYZsq{}} \PY{c+c1}{\PYZsh{} Als je een databank lokaal draait, is dit localhost.}
        \PY{n}{db} \PY{o}{=} \PY{l+s+s1}{\PYZsq{}}\PY{l+s+s1}{lahman2016ZonderIndexen}\PY{l+s+s1}{\PYZsq{}}      \PY{c+c1}{\PYZsh{} Naam van de gegevensbank op je XAMPP Mysql server}
        
        \PY{c+c1}{\PYZsh{} We verbinden met de gegevensbank}
        \PY{n}{c} \PY{o}{=} \PY{n}{verbind\PYZus{}met\PYZus{}GB}\PY{p}{(}\PY{n}{username}\PY{p}{,} \PY{n}{hostname}\PY{p}{,} \PY{n}{db}\PY{p}{)}
\end{Verbatim}


    \begin{Verbatim}[commandchars=\\\{\}]
········

    \end{Verbatim}

    In onderstaande cel bekomen we de runtime van de originele query door
deze vijf keer te runnen, om tot een betere schatting te komen. Deze
conventie van runtimes meten dient behouden te worden.

    \begin{Verbatim}[commandchars=\\\{\}]
{\color{incolor}In [{\color{incolor}7}]:} \PY{n}{timings}\PY{o}{=}\PY{p}{[}\PY{p}{]}
        
        \PY{c+c1}{\PYZsh{} We voeren deze query 5 keer uit om een betere schatting te krijgen van de effectieve runtime.}
        \PY{k}{for} \PY{n}{i} \PY{o+ow}{in} \PY{n+nb}{range}\PY{p}{(}\PY{l+m+mi}{5}\PY{p}{)}\PY{p}{:}
            \PY{n}{t1} \PY{o}{=} \PY{n}{time}\PY{o}{.}\PY{n}{time}\PY{p}{(}\PY{p}{)} \PY{c+c1}{\PYZsh{} Start time}
            \PY{n}{df\PYZus{}orig} \PY{o}{=} \PY{n}{query\PYZus{}to\PYZus{}optimize}\PY{p}{(}\PY{n}{c}\PY{p}{,} \PY{n}{column\PYZus{}names}\PY{p}{,} \PY{n}{jaar\PYZus{}1}\PY{o}{=}\PY{l+m+mi}{2000}\PY{p}{,} \PY{n}{jaar\PYZus{}2}\PY{o}{=}\PY{l+m+mi}{1990}\PY{p}{,} \PY{n}{lengte}\PY{o}{=}\PY{l+m+mi}{75}\PY{p}{)}
            \PY{n}{t2} \PY{o}{=} \PY{n}{time}\PY{o}{.}\PY{n}{time}\PY{p}{(}\PY{p}{)} \PY{c+c1}{\PYZsh{} Stop time}
            \PY{n}{timings}\PY{o}{.}\PY{n}{append}\PY{p}{(}\PY{n}{t2}\PY{o}{\PYZhy{}}\PY{n}{t1}\PY{p}{)} 
            
        \PY{n}{orig\PYZus{}runtime}\PY{o}{=}\PY{n}{np}\PY{o}{.}\PY{n}{mean}\PY{p}{(}\PY{n}{timings}\PY{p}{)} \PY{c+c1}{\PYZsh{} De uiteindelijke runtime is een gemiddelde van 5 runs.}
        
        \PY{n+nb}{print}\PY{p}{(}\PY{l+s+s1}{\PYZsq{}}\PY{l+s+s1}{De originele query duurt: }\PY{l+s+s1}{\PYZsq{}}\PY{p}{,} \PY{l+s+s2}{\PYZdq{}}\PY{l+s+si}{\PYZob{}0:.2f\PYZcb{}}\PY{l+s+s2}{\PYZdq{}}\PY{o}{.}\PY{n}{format}\PY{p}{(}\PY{n}{orig\PYZus{}runtime}\PY{p}{)}\PY{p}{,}\PY{l+s+s1}{\PYZsq{}}\PY{l+s+s1}{ seconden}\PY{l+s+s1}{\PYZsq{}}\PY{p}{)}
\end{Verbatim}


    \begin{Verbatim}[commandchars=\\\{\}]
De originele query duurt:  1.41  seconden

    \end{Verbatim}

    \subsubsection{Optimalisatie}\label{optimalisatie}

In deze sectie is het de bedoeling dat je je optimalisatie realiseert.
Wat hier al ingevuld staat qua code is enkel om je op weg te helpen.

Je mag zoveel codecellen toevoegen als je nodig acht om je verhaal te
vertellen. Leg uit wat je doet, en waarom je dat doet. Denk hierbij
zeker aan het EXPLAIN statement zoals gezien in de les. Gebruik EXPLAIN
om te identificeren hoe MySQL de query omzet en kijk zo welke effecten
je verbeteringen hebben. Leg de resultaten van EXPLAIN kort uit.

Beschouw de uitleg rond je oplossing als een soort mini-verslag dat je
code en query uitlegt.

    \begin{longtable}[]{@{}l@{}}
\toprule
\begin{minipage}[b]{0.35\columnwidth}\raggedright\strut
Verklaring bij gebruikte indexen:\strut
\end{minipage}\tabularnewline
\midrule
\endhead
\begin{minipage}[t]{0.35\columnwidth}\raggedright\strut
In de database zijn de volgende indexen geplaatst
\_\_\_\_\_\_\_\_\_\_\_\_\_\_\_\_\_\_\_\_\_\_\_\_\_\_\_\_\_\_\_\_ indexen
op selecties:\strut
\end{minipage}\tabularnewline
\begin{minipage}[t]{0.35\columnwidth}\raggedright\strut
1. Een B+ boom op de kolom yearID van de table halloffame aangezien daar
een rangequery uitgewerkt staat en een interessante eigenschap van B+
bomen is dat aan de bladeren wijzers staan die wijzen naar het volgende
blad.\strut
\end{minipage}\tabularnewline
\begin{minipage}[t]{0.35\columnwidth}\raggedright\strut
de database bezit informatie tussen 1975 en 2011 dit betekent dat er 36
verschillende waarden zijn voor yearID specifiek in dit geval zijn er 21
waarden die voldoen aan de ongelijkheid dit betekent dat er over
(21/36)*4200 =2450 tuples moet itereren.\strut
\end{minipage}\tabularnewline
\begin{minipage}[t]{0.35\columnwidth}\raggedright\strut
2. Een hash index op de kolommen yearID van de tabellen Pitching en
Batting aangezien daar een point query uitgewerkt is. -\/-\textgreater{}
meestal 1 à 2 blokken\strut
\end{minipage}\tabularnewline
\begin{minipage}[t]{0.35\columnwidth}\raggedright\strut
3. Een B+ boom op de kolommen HR, height, weight en SV van de tabellen
batting, master en pitching. Aangezien er over deze kolommen AVG genomen
is, wat neerkomt op SUM en COUNT en dit kan zeer efficient mbv B+bomen
door de bladwijzers die aanwezig zijn op bladeren van de boom.\strut
\end{minipage}\tabularnewline
\begin{minipage}[t]{0.35\columnwidth}\raggedright\strut
4. Gebruik maken van b+ boom voor BirthState kolom bij de tabel master
omdat men het resultaat groepeert volgens de waarden van deze kolom.
\_\_\_\_\_\_\_\_\_\_\_\_\_\_\_\_\_\_\_\_\_\_\_\_\_\_\_\_\_\_\_\_ indexen
op joins:\strut
\end{minipage}\tabularnewline
\begin{minipage}[t]{0.35\columnwidth}\raggedright\strut
Een clustering index op de kolom playerID van master,batting, pitching
en halloffame\strut
\end{minipage}\tabularnewline
\begin{minipage}[t]{0.35\columnwidth}\raggedright\strut
1. -\/-\textgreater{}master: primary index op playerID(=unieke kolom)
=\textgreater{} loopt over index 2. -\/-\textgreater{}batting: 103.000
tupels met 19.000 unieke waarden voor playerID =\textgreater{} 6 tupels
per blok 3. -\/-\textgreater{}pitching: 45.000 tuples met 9.300 unieke
waarden voor playerID =\textgreater{} 5 tupels per blok 4.
-\/-\textgreater{}halloffame: 4.200 tupels met 1.260 unieke waarden voor
playerID =\textgreater{} 2 tupels per blok\strut
\end{minipage}\tabularnewline
\bottomrule
\end{longtable}

    \begin{Verbatim}[commandchars=\\\{\}]
{\color{incolor}In [{\color{incolor}8}]:} \PY{k}{def} \PY{n+nf}{optim\PYZus{}enkel\PYZus{}indexen}\PY{p}{(}\PY{n}{connection}\PY{p}{,} \PY{n}{column\PYZus{}names}\PY{p}{,} \PY{n}{jaar\PYZus{}1}\PY{o}{=}\PY{l+m+mi}{2000}\PY{p}{,} \PY{n}{jaar\PYZus{}2}\PY{o}{=}\PY{l+m+mi}{1990}\PY{p}{,} \PY{n}{lengte}\PY{o}{=}\PY{l+m+mi}{75}\PY{p}{)}\PY{p}{:}
            \PY{n}{query}\PY{o}{=}\PY{l+s+s2}{\PYZdq{}\PYZdq{}\PYZdq{}}
        \PY{l+s+s2}{    SELECT m.birthState, AVG(m.weight), AVG(m.height), AVG(bat.HR), AVG(pit.SV)}
        \PY{l+s+s2}{    FROM Master AS m,}
        \PY{l+s+s2}{        Pitching AS pit,}
        \PY{l+s+s2}{        Batting AS bat,}
        \PY{l+s+s2}{        HallOfFame AS hof}
        \PY{l+s+s2}{    WHERE pit.yearID = }\PY{l+s+si}{\PYZob{}\PYZcb{}}
        \PY{l+s+s2}{        AND bat.yearID = }\PY{l+s+si}{\PYZob{}\PYZcb{}}
        \PY{l+s+s2}{        AND pit.playerID = m.playerID}
        \PY{l+s+s2}{        AND bat.playerID = m.playerID}
        \PY{l+s+s2}{        AND m.playerID = hof.playerID}
        \PY{l+s+s2}{        AND hof.yearID \PYZgt{} }\PY{l+s+si}{\PYZob{}\PYZcb{}}
        \PY{l+s+s2}{    GROUP BY m.birthState}
        \PY{l+s+s2}{    HAVING AVG(m.height) \PYZgt{} }\PY{l+s+si}{\PYZob{}\PYZcb{}}
        \PY{l+s+s2}{    ORDER BY m.birthState ASC;}
        \PY{l+s+s2}{    }\PY{l+s+s2}{\PYZdq{}\PYZdq{}\PYZdq{}}\PY{o}{.}\PY{n}{format}\PY{p}{(}\PY{n}{jaar\PYZus{}1}\PY{p}{,} \PY{n}{jaar\PYZus{}1}\PY{p}{,} \PY{n}{jaar\PYZus{}2}\PY{p}{,} \PY{n}{lengte}\PY{p}{)}
\end{Verbatim}


    \begin{Verbatim}[commandchars=\\\{\}]
{\color{incolor}In [{\color{incolor}9}]:} \PY{n}{username} \PY{o}{=} \PY{l+s+s1}{\PYZsq{}}\PY{l+s+s1}{root}\PY{l+s+s1}{\PYZsq{}}      \PY{c+c1}{\PYZsh{} Vervang dit als je via een andere user queries stuurt}
        \PY{n}{hostname} \PY{o}{=} \PY{l+s+s1}{\PYZsq{}}\PY{l+s+s1}{localhost}\PY{l+s+s1}{\PYZsq{}} \PY{c+c1}{\PYZsh{} Als je een databank lokaal draait, is dit localhost.}
        \PY{n}{db} \PY{o}{=} \PY{l+s+s1}{\PYZsq{}}\PY{l+s+s1}{lahman2016test}\PY{l+s+s1}{\PYZsq{}}      \PY{c+c1}{\PYZsh{} Naam van de gegevensbank op je XAMPP Mysql server}
        
        \PY{c+c1}{\PYZsh{} We verbinden met de gegevensbank}
        \PY{n}{c} \PY{o}{=} \PY{n}{verbind\PYZus{}met\PYZus{}GB}\PY{p}{(}\PY{n}{username}\PY{p}{,} \PY{n}{hostname}\PY{p}{,} \PY{n}{db}\PY{p}{)}
\end{Verbatim}


    \begin{Verbatim}[commandchars=\\\{\}]
········

    \end{Verbatim}

    \begin{Verbatim}[commandchars=\\\{\}]
{\color{incolor}In [{\color{incolor}10}]:} \PY{n}{timings3}\PY{o}{=}\PY{p}{[}\PY{p}{]}
         
         \PY{c+c1}{\PYZsh{} We voeren deze query 5 keer uit om een betere schatting te krijgen van de effectieve runtime.}
         \PY{k}{for} \PY{n}{i} \PY{o+ow}{in} \PY{n+nb}{range}\PY{p}{(}\PY{l+m+mi}{5}\PY{p}{)}\PY{p}{:}
             \PY{n}{t1} \PY{o}{=} \PY{n}{time}\PY{o}{.}\PY{n}{time}\PY{p}{(}\PY{p}{)} \PY{c+c1}{\PYZsh{} Start time}
             \PY{n}{df\PYZus{}orig} \PY{o}{=} \PY{n}{query\PYZus{}to\PYZus{}optimize}\PY{p}{(}\PY{n}{c}\PY{p}{,} \PY{n}{column\PYZus{}names}\PY{p}{,} \PY{n}{jaar\PYZus{}1}\PY{o}{=}\PY{l+m+mi}{2000}\PY{p}{,} \PY{n}{jaar\PYZus{}2}\PY{o}{=}\PY{l+m+mi}{1990}\PY{p}{,} \PY{n}{lengte}\PY{o}{=}\PY{l+m+mi}{75}\PY{p}{)}
             \PY{n}{t2} \PY{o}{=} \PY{n}{time}\PY{o}{.}\PY{n}{time}\PY{p}{(}\PY{p}{)} \PY{c+c1}{\PYZsh{} Stop time}
             \PY{n}{timings3}\PY{o}{.}\PY{n}{append}\PY{p}{(}\PY{n}{t2}\PY{o}{\PYZhy{}}\PY{n}{t1}\PY{p}{)} 
             
         \PY{n}{orig\PYZus{}runtime}\PY{o}{=}\PY{n}{np}\PY{o}{.}\PY{n}{mean}\PY{p}{(}\PY{n}{timings3}\PY{p}{)} \PY{c+c1}{\PYZsh{} De uiteindelijke runtime is een gemiddelde van 5 runs.}
         
         \PY{n+nb}{print}\PY{p}{(}\PY{l+s+s1}{\PYZsq{}}\PY{l+s+s1}{De originele query duurt: }\PY{l+s+s1}{\PYZsq{}}\PY{p}{,} \PY{l+s+s2}{\PYZdq{}}\PY{l+s+si}{\PYZob{}0:.2f\PYZcb{}}\PY{l+s+s2}{\PYZdq{}}\PY{o}{.}\PY{n}{format}\PY{p}{(}\PY{n}{orig\PYZus{}runtime}\PY{p}{)}\PY{p}{,}\PY{l+s+s1}{\PYZsq{}}\PY{l+s+s1}{ seconden}\PY{l+s+s1}{\PYZsq{}}\PY{p}{)}
\end{Verbatim}


    \begin{Verbatim}[commandchars=\\\{\}]
De originele query duurt:  0.04  seconden

    \end{Verbatim}

    \begin{Verbatim}[commandchars=\\\{\}]
{\color{incolor}In [{\color{incolor}11}]:} \PY{c+c1}{\PYZsh{} Maak hier je eerste optimalisatie}
         \PY{k}{def} \PY{n+nf}{optim\PYZus{}1}\PY{p}{(}\PY{n}{connection}\PY{p}{,} \PY{n}{column\PYZus{}names}\PY{p}{,} \PY{n}{jaar\PYZus{}1}\PY{o}{=}\PY{l+m+mi}{2000}\PY{p}{,} \PY{n}{jaar\PYZus{}2}\PY{o}{=}\PY{l+m+mi}{1990}\PY{p}{,} \PY{n}{lengte}\PY{o}{=}\PY{l+m+mi}{75}\PY{p}{)}\PY{p}{:}
             \PY{c+c1}{\PYZsh{} het opstellen }
             \PY{n}{query}\PY{o}{=}\PY{l+s+s2}{\PYZdq{}\PYZdq{}\PYZdq{}}
         \PY{l+s+s2}{   }
         \PY{l+s+s2}{     SELECT bpm.BirthState, AVG(bpm.weight), AVG(bpm.height), AVG(bpm.HR), AVG(bpm.SV)}
         \PY{l+s+s2}{     FROM}
         \PY{l+s+s2}{        (SELECT m.playerID, m.BirthState, m.weight, m.height, bp.SV, bp.HR}
         \PY{l+s+s2}{         FROM   }
         \PY{l+s+s2}{            (SELECT p.playerID, p.SV, b.HR}
         \PY{l+s+s2}{             FROM   (SELECT bat.playerID, bat.yearID, bat.HR}
         \PY{l+s+s2}{                     FROM batting as bat}
         \PY{l+s+s2}{                    ) as b, }
         \PY{l+s+s2}{                }
         \PY{l+s+s2}{                    (SELECT pit.yearID, pit.playerID, pit.SV}
         \PY{l+s+s2}{                     FROM pitching as pit}
         \PY{l+s+s2}{                    ) as p}
         \PY{l+s+s2}{                }
         \PY{l+s+s2}{             WHERE     p.yearID = }\PY{l+s+si}{\PYZob{}\PYZcb{}}
         \PY{l+s+s2}{                   AND b.yearID = }\PY{l+s+si}{\PYZob{}\PYZcb{}}\PY{l+s+s2}{   }
         \PY{l+s+s2}{                   AND p.playerID = b.playerID}
         \PY{l+s+s2}{            ) as bp, }
         \PY{l+s+s2}{        }
         \PY{l+s+s2}{        }
         \PY{l+s+s2}{            (SELECT mas.playerID, mas.BirthState, mas.weight, mas.height }
         \PY{l+s+s2}{             FROM master as mas}
         \PY{l+s+s2}{            ) as m}
         \PY{l+s+s2}{        }
         \PY{l+s+s2}{        WHERE }
         \PY{l+s+s2}{            bp.playerID = m.playerID}
         \PY{l+s+s2}{       ) as bpm,}
         \PY{l+s+s2}{       }
         \PY{l+s+s2}{       }
         \PY{l+s+s2}{        (SELECT hof.playerID}
         \PY{l+s+s2}{         FROM   halloffame as hof}
         \PY{l+s+s2}{         WHERE hof.yearID \PYZgt{}}\PY{l+s+si}{\PYZob{}\PYZcb{}}
         \PY{l+s+s2}{        ) as h}
         \PY{l+s+s2}{    WHERE h.playerID = bpm.playerID}
         \PY{l+s+s2}{       }
         \PY{l+s+s2}{    GROUP BY bpm.birthState}
         \PY{l+s+s2}{    HAVING AVG(bpm.height) \PYZgt{} }\PY{l+s+si}{\PYZob{}\PYZcb{}}
         \PY{l+s+s2}{    ORDER BY bpm.birthState ASC;}
         \PY{l+s+s2}{   }
         \PY{l+s+s2}{   }
         \PY{l+s+s2}{    }\PY{l+s+s2}{\PYZdq{}\PYZdq{}\PYZdq{}}\PY{o}{.}\PY{n}{format}\PY{p}{(}\PY{n}{jaar\PYZus{}1}\PY{p}{,} \PY{n}{jaar\PYZus{}1}\PY{p}{,} \PY{n}{jaar\PYZus{}2}\PY{p}{,} \PY{n}{lengte}\PY{p}{)}
         
             
             \PY{c+c1}{\PYZsh{} Stap 2 \PYZam{} 3}
             \PY{n}{res} \PY{o}{=} \PY{n}{run\PYZus{}query}\PY{p}{(}\PY{n}{connection}\PY{p}{,} \PY{n}{query}\PY{p}{)}         \PY{c+c1}{\PYZsh{} Query uitvoeren}
             \PY{n}{df} \PY{o}{=} \PY{n}{res\PYZus{}to\PYZus{}df}\PY{p}{(}\PY{n}{res}\PY{p}{,} \PY{n}{column\PYZus{}names}\PY{p}{)}          \PY{c+c1}{\PYZsh{} Query in DataFrame brengen}
             
             \PY{k}{return} \PY{n}{df}
\end{Verbatim}


    \begin{Verbatim}[commandchars=\\\{\}]
{\color{incolor}In [{\color{incolor}12}]:} \PY{n}{username} \PY{o}{=} \PY{l+s+s1}{\PYZsq{}}\PY{l+s+s1}{root}\PY{l+s+s1}{\PYZsq{}}      \PY{c+c1}{\PYZsh{} Vervang dit als je via een andere user queries stuurt}
         \PY{n}{hostname} \PY{o}{=} \PY{l+s+s1}{\PYZsq{}}\PY{l+s+s1}{localhost}\PY{l+s+s1}{\PYZsq{}} \PY{c+c1}{\PYZsh{} Als je een databank lokaal draait, is dit localhost.}
         \PY{n}{db} \PY{o}{=} \PY{l+s+s1}{\PYZsq{}}\PY{l+s+s1}{lahman2016test}\PY{l+s+s1}{\PYZsq{}}      \PY{c+c1}{\PYZsh{} Naam van de gegevensbank op je XAMPP Mysql server}
         
         \PY{c+c1}{\PYZsh{} We verbinden met de gegevensbank}
         \PY{n}{c} \PY{o}{=} \PY{n}{verbind\PYZus{}met\PYZus{}GB}\PY{p}{(}\PY{n}{username}\PY{p}{,} \PY{n}{hostname}\PY{p}{,} \PY{n}{db}\PY{p}{)}
\end{Verbatim}


    \begin{Verbatim}[commandchars=\\\{\}]
········

    \end{Verbatim}

    \begin{Verbatim}[commandchars=\\\{\}]
{\color{incolor}In [{\color{incolor}13}]:} \PY{n}{timings2}\PY{o}{=}\PY{p}{[}\PY{p}{]}
         
         \PY{c+c1}{\PYZsh{} We voeren deze query 5 keer uit om een betere schatting te krijgen van de effectieve runtime.}
         \PY{k}{for} \PY{n}{i} \PY{o+ow}{in} \PY{n+nb}{range}\PY{p}{(}\PY{l+m+mi}{5}\PY{p}{)}\PY{p}{:}
             \PY{n}{t1} \PY{o}{=} \PY{n}{time}\PY{o}{.}\PY{n}{time}\PY{p}{(}\PY{p}{)} \PY{c+c1}{\PYZsh{} Start time}
             \PY{n}{df\PYZus{}orig} \PY{o}{=} \PY{n}{query\PYZus{}to\PYZus{}optimize}\PY{p}{(}\PY{n}{c}\PY{p}{,} \PY{n}{column\PYZus{}names}\PY{p}{,} \PY{n}{jaar\PYZus{}1}\PY{o}{=}\PY{l+m+mi}{2000}\PY{p}{,} \PY{n}{jaar\PYZus{}2}\PY{o}{=}\PY{l+m+mi}{1990}\PY{p}{,} \PY{n}{lengte}\PY{o}{=}\PY{l+m+mi}{75}\PY{p}{)}
             \PY{n}{t2} \PY{o}{=} \PY{n}{time}\PY{o}{.}\PY{n}{time}\PY{p}{(}\PY{p}{)} \PY{c+c1}{\PYZsh{} Stop time}
             \PY{n}{timings2}\PY{o}{.}\PY{n}{append}\PY{p}{(}\PY{n}{t2}\PY{o}{\PYZhy{}}\PY{n}{t1}\PY{p}{)} 
             
         \PY{n}{orig\PYZus{}runtime}\PY{o}{=}\PY{n}{np}\PY{o}{.}\PY{n}{mean}\PY{p}{(}\PY{n}{timings2}\PY{p}{)} \PY{c+c1}{\PYZsh{} De uiteindelijke runtime is een gemiddelde van 5 runs.}
         
         \PY{n+nb}{print}\PY{p}{(}\PY{l+s+s1}{\PYZsq{}}\PY{l+s+s1}{De originele query duurt: }\PY{l+s+s1}{\PYZsq{}}\PY{p}{,} \PY{l+s+s2}{\PYZdq{}}\PY{l+s+si}{\PYZob{}0:.2f\PYZcb{}}\PY{l+s+s2}{\PYZdq{}}\PY{o}{.}\PY{n}{format}\PY{p}{(}\PY{n}{orig\PYZus{}runtime}\PY{p}{)}\PY{p}{,}\PY{l+s+s1}{\PYZsq{}}\PY{l+s+s1}{ seconden}\PY{l+s+s1}{\PYZsq{}}\PY{p}{)}
\end{Verbatim}


    \begin{Verbatim}[commandchars=\\\{\}]
De originele query duurt:  0.02  seconden

    \end{Verbatim}

    \subsubsection{Controle}\label{controle}

De onderstaande codecel vergelijkt de runtime van de optimalisatie met
die van de originele query, door een gemiddelde te nemen over 5 runs.
Ook wordt er nagegaan dat de resultaten zeker volledig gelijk zijn.
Wijzig deze controlewijze niet.

    \begin{Verbatim}[commandchars=\\\{\}]
{\color{incolor}In [{\color{incolor}14}]:} \PY{c+c1}{\PYZsh{} Test hier je eerste optimalisatie}
         
         \PY{n}{timings\PYZus{}optim\PYZus{}1} \PY{o}{=} \PY{p}{[}\PY{p}{]}
         \PY{k}{for} \PY{n}{i} \PY{o+ow}{in} \PY{n+nb}{range}\PY{p}{(}\PY{l+m+mi}{5}\PY{p}{)}\PY{p}{:}
             \PY{n}{t1} \PY{o}{=} \PY{n}{time}\PY{o}{.}\PY{n}{time}\PY{p}{(}\PY{p}{)} \PY{c+c1}{\PYZsh{} Start time}
             \PY{n}{df\PYZus{}optim\PYZus{}1} \PY{o}{=} \PY{n}{optim\PYZus{}1}\PY{p}{(}\PY{n}{c}\PY{p}{,} \PY{n}{column\PYZus{}names}\PY{p}{,} \PY{n}{jaar\PYZus{}1}\PY{o}{=}\PY{l+m+mi}{2000}\PY{p}{,} \PY{n}{jaar\PYZus{}2}\PY{o}{=}\PY{l+m+mi}{1990}\PY{p}{,} \PY{n}{lengte}\PY{o}{=}\PY{l+m+mi}{75}\PY{p}{)}
             \PY{n}{t2} \PY{o}{=} \PY{n}{time}\PY{o}{.}\PY{n}{time}\PY{p}{(}\PY{p}{)} \PY{c+c1}{\PYZsh{} Stop time}
             \PY{n}{timings\PYZus{}optim\PYZus{}1}\PY{o}{.}\PY{n}{append}\PY{p}{(}\PY{n}{t2}\PY{o}{\PYZhy{}}\PY{n}{t1}\PY{p}{)} 
             
         \PY{n}{optim\PYZus{}1\PYZus{}runtime} \PY{o}{=} \PY{n}{np}\PY{o}{.}\PY{n}{mean}\PY{p}{(}\PY{n}{timings\PYZus{}optim\PYZus{}1}\PY{p}{)} \PY{c+c1}{\PYZsh{} Runtime optimalisatie 1}
         
         \PY{c+c1}{\PYZsh{} Vergelijken met originele query}
         \PY{n}{diff} \PY{o}{=} \PY{n}{orig\PYZus{}runtime}\PY{o}{\PYZhy{}}\PY{n}{optim\PYZus{}1\PYZus{}runtime} \PY{c+c1}{\PYZsh{} Winst t.o.v. origineel}
         \PY{n}{rel\PYZus{}diff} \PY{o}{=} \PY{n}{diff}\PY{o}{/}\PY{n}{orig\PYZus{}runtime} \PY{c+c1}{\PYZsh{} Relatieve winst}
         \PY{n}{check} \PY{o}{=} \PY{n}{check\PYZus{}perfect\PYZus{}match}\PY{p}{(}\PY{n}{df\PYZus{}orig}\PY{p}{,}\PY{n}{df\PYZus{}optim\PYZus{}1}\PY{p}{)} \PY{c+c1}{\PYZsh{} Nagaan of de resultaten exact gelijk zijn.}
         
         \PY{c+c1}{\PYZsh{} Belangrijkste resultaten printen.}
         \PY{n+nb}{print}\PY{p}{(}\PY{l+s+s1}{\PYZsq{}}\PY{l+s+s1}{De optimalisatie duurt: }\PY{l+s+s1}{\PYZsq{}}\PY{p}{,} \PY{l+s+s2}{\PYZdq{}}\PY{l+s+si}{\PYZob{}0:.2f\PYZcb{}}\PY{l+s+s2}{\PYZdq{}}\PY{o}{.}\PY{n}{format}\PY{p}{(}\PY{n}{optim\PYZus{}1\PYZus{}runtime}\PY{p}{)}\PY{p}{,}\PY{l+s+s1}{\PYZsq{}}\PY{l+s+s1}{ seconden}\PY{l+s+s1}{\PYZsq{}}\PY{p}{)}
         \PY{n+nb}{print}\PY{p}{(}\PY{l+s+s1}{\PYZsq{}}\PY{l+s+s1}{De originele versie duurde: }\PY{l+s+s1}{\PYZsq{}}\PY{p}{,} \PY{l+s+s2}{\PYZdq{}}\PY{l+s+si}{\PYZob{}0:.2f\PYZcb{}}\PY{l+s+s2}{\PYZdq{}}\PY{o}{.}\PY{n}{format}\PY{p}{(}\PY{n}{orig\PYZus{}runtime}\PY{p}{)}\PY{p}{,}\PY{l+s+s1}{\PYZsq{}}\PY{l+s+s1}{ seconden}\PY{l+s+s1}{\PYZsq{}}\PY{p}{)}
         \PY{n+nb}{print}\PY{p}{(}\PY{l+s+s1}{\PYZsq{}}\PY{l+s+s1}{De netto tijdwinst is dus: }\PY{l+s+s1}{\PYZsq{}}\PY{p}{,}\PY{l+s+s2}{\PYZdq{}}\PY{l+s+si}{\PYZob{}0:.2f\PYZcb{}}\PY{l+s+s2}{\PYZdq{}}\PY{o}{.}\PY{n}{format}\PY{p}{(}\PY{n}{diff}\PY{p}{)}\PY{p}{,}\PY{l+s+s1}{\PYZsq{}}\PY{l+s+s1}{ seconden, oftewel}\PY{l+s+s1}{\PYZsq{}}\PY{p}{,} \PY{l+s+s2}{\PYZdq{}}\PY{l+s+si}{\PYZob{}0:.2f\PYZcb{}}\PY{l+s+s2}{\PYZdq{}}\PY{o}{.}\PY{n}{format}\PY{p}{(}\PY{l+m+mi}{100}\PY{o}{*}\PY{n}{rel\PYZus{}diff}\PY{p}{)}\PY{p}{,} \PY{l+s+s1}{\PYZsq{}}\PY{l+s+s1}{\PYZpc{}}\PY{l+s+s1}{\PYZsq{}} \PY{p}{)}
         \PY{n+nb}{print}\PY{p}{(}\PY{l+s+s1}{\PYZsq{}}\PY{l+s+s1}{Is het resultaat equivalent? }\PY{l+s+s1}{\PYZsq{}}\PY{p}{,} \PY{n}{check}\PY{p}{)}
\end{Verbatim}


    \begin{Verbatim}[commandchars=\\\{\}]
De optimalisatie duurt:  0.01  seconden
De originele versie duurde:  0.02  seconden
De netto tijdwinst is dus:  0.01  seconden, oftewel 59.84 \%
Is het resultaat equivalent?  True

    \end{Verbatim}

    \subsection{Visualisatie}\label{visualisatie}

\subsubsection{Opgave}\label{opgave}

In het laatste onderdeel van het werkje laten we jullie volledig vrij.
Jullie hebben ondertussen geleerd om data uit een gegevensbank op te
halen, de volgende stap is om deze data effectief voor iets nuttigs te
gebruiken.

Een snelle manier om gegevens overzichtelijk weer te geven is
visualisatie. Een goede visualisatie kan zeker in een eerste fase van
data-analyse echt al inzichten verschaffen. Wij vragen dus aan jullie
een of meerdere visualisaties te realiseren, met data uit de
gegevensbank die jullie voor dit werkje gebruikt hebben.

Een goede oplossing van deze opgave voldoet aan volgende criteria: *
Schrijf een query die informatie uit de gegevensbank ophaalt, en in een
DataFrame stopt. * Maak een interessante visualisatie van (een deel van)
de data in die DataFrame * Je beschrijft kort (mini-verslag hier in de
notebook) wat je doet, en wat de inzichten zijn die de visualisatie
verschaft. * Je bent in staat om dit mini-verslag ook mondeling te
verdedigen, i.e. antwoorden op: wat doe je, waarom doe je dit, en wat
was het resultaat. * 1 visualisatie volstaat, meer mag altijd.

\subsubsection{Nuttige packages}\label{nuttige-packages}

Om jullie al een beetje op weg te helpen, geven we jullie alvast de naam
van twee handige en veelvoorkomende visualisatietools die goed werken
met pandas DataFrames:

\begin{enumerate}
\def\labelenumi{\arabic{enumi}.}
\tightlist
\item
  seaborn
\item
  matplotlib
\end{enumerate}

Voor een korte inleiding ivm seaborn kan je hier kijken:
https://chrisalbon.com/python/pandas\_with\_seaborn.html Voor een korte
inleiding ivm matplotlib kan je hier kijken:
http://pandas.pydata.org/pandas-docs/stable/visualization.html

Natuurlijk moet je vooraleer je aan zo'n visualisatie kan beginnen, al
een pandas DataFrame hebben dat de juiste informatie bevat. De eerste
stap zal er dus in bestaan om een goede query te schrijven die de nodige
informatie uit de gegevensbank weet te halen, gevolgd door een
visualisatie van de DataFrame die uit die query volgt.

    \begin{longtable}[]{@{}l@{}}
\toprule
Plot 1: evolutie aantal spelers\tabularnewline
\bottomrule
\end{longtable}

De volgende grafiek toont het aantal spelers dat elk jaar deelnam aan
een baseballevenement. Er is een stijgende trend op te merken in de
grafiek.

    \begin{Verbatim}[commandchars=\\\{\}]
{\color{incolor}In [{\color{incolor}31}]:} \PY{n}{column\PYZus{}names\PYZus{}vis} \PY{o}{=} \PY{p}{[}\PY{l+s+s1}{\PYZsq{}}\PY{l+s+s1}{players}\PY{l+s+s1}{\PYZsq{}}\PY{p}{,}\PY{l+s+s1}{\PYZsq{}}\PY{l+s+s1}{year}\PY{l+s+s1}{\PYZsq{}}\PY{p}{]}
\end{Verbatim}


    \begin{Verbatim}[commandchars=\\\{\}]
{\color{incolor}In [{\color{incolor}39}]:} \PY{c+c1}{\PYZsh{} Maak hier je eerste optimalisatie}
         \PY{k}{def} \PY{n+nf}{query\PYZus{}visualisatie}\PY{p}{(}\PY{n}{connection}\PY{p}{,} \PY{n}{column\PYZus{}names}\PY{p}{)}\PY{p}{:}
             \PY{c+c1}{\PYZsh{} het opstellen }
             \PY{n}{query}\PY{o}{=}\PY{l+s+s2}{\PYZdq{}\PYZdq{}\PYZdq{}}
         \PY{l+s+s2}{        SELECT count(DISTINCT a.playerID), a.yearID}
         \PY{l+s+s2}{        FROM   appearances as a}
         \PY{l+s+s2}{        GROUP BY a.yearID}
         \PY{l+s+s2}{        }\PY{l+s+s2}{\PYZdq{}\PYZdq{}\PYZdq{}}
             \PY{c+c1}{\PYZsh{} Stap 2 \PYZam{} 3}
             \PY{n}{res} \PY{o}{=} \PY{n}{run\PYZus{}query}\PY{p}{(}\PY{n}{connection}\PY{p}{,} \PY{n}{query}\PY{p}{)}         \PY{c+c1}{\PYZsh{} Query uitvoeren}
             \PY{n}{df} \PY{o}{=} \PY{n}{res\PYZus{}to\PYZus{}df}\PY{p}{(}\PY{n}{res}\PY{p}{,} \PY{n}{column\PYZus{}names}\PY{p}{)}          \PY{c+c1}{\PYZsh{} Query in DataFrame brengen}
                
             \PY{k}{return} \PY{n}{df}
\end{Verbatim}


    \begin{Verbatim}[commandchars=\\\{\}]
{\color{incolor}In [{\color{incolor}40}]:} \PY{n}{username} \PY{o}{=} \PY{l+s+s1}{\PYZsq{}}\PY{l+s+s1}{root}\PY{l+s+s1}{\PYZsq{}}      \PY{c+c1}{\PYZsh{} Vervang dit als je via een andere user queries stuurt}
         \PY{n}{hostname} \PY{o}{=} \PY{l+s+s1}{\PYZsq{}}\PY{l+s+s1}{localhost}\PY{l+s+s1}{\PYZsq{}} \PY{c+c1}{\PYZsh{} Als je een databank lokaal draait, is dit localhost.}
         \PY{n}{db} \PY{o}{=} \PY{l+s+s1}{\PYZsq{}}\PY{l+s+s1}{lahman2016test}\PY{l+s+s1}{\PYZsq{}}      \PY{c+c1}{\PYZsh{} Naam van de gegevensbank op je XAMPP Mysql server}
         
         \PY{c+c1}{\PYZsh{} We verbinden met de gegevensbank}
         \PY{n}{c} \PY{o}{=} \PY{n}{verbind\PYZus{}met\PYZus{}GB}\PY{p}{(}\PY{n}{username}\PY{p}{,} \PY{n}{hostname}\PY{p}{,} \PY{n}{db}\PY{p}{)}
\end{Verbatim}


    \begin{Verbatim}[commandchars=\\\{\}]
········

    \end{Verbatim}

    \begin{Verbatim}[commandchars=\\\{\}]
{\color{incolor}In [{\color{incolor}44}]:} \PY{c+c1}{\PYZsh{} De voorbeeldquery heeft dezelfde kolomnamen als query 1, dus we gebruiken die}
         \PY{n}{kolommen} \PY{o}{=} \PY{n}{column\PYZus{}names\PYZus{}vis}
         
         \PY{c+c1}{\PYZsh{} Functie uitvoeren, geeft resultaat van de query in een DataFrame}
         \PY{n}{df} \PY{o}{=} \PY{n}{query\PYZus{}visualisatie}\PY{p}{(}\PY{n}{c}\PY{p}{,} \PY{n}{column\PYZus{}names\PYZus{}vis}\PY{p}{)}
         \PY{n}{plt}\PY{o}{.}\PY{n}{plot}\PY{p}{(}\PY{n}{df}\PY{p}{[}\PY{l+s+s1}{\PYZsq{}}\PY{l+s+s1}{players}\PY{l+s+s1}{\PYZsq{}}\PY{p}{]}\PY{p}{)}
         \PY{c+c1}{\PYZsh{} We inspecteren de eerste paar resultaten (voor alles te zien: laat .head() weg)}
         \PY{n}{df}\PY{o}{.}\PY{n}{head}\PY{p}{(}\PY{p}{)}
\end{Verbatim}


\begin{Verbatim}[commandchars=\\\{\}]
{\color{outcolor}Out[{\color{outcolor}44}]:}    players  year
         0      115  1871
         1      143  1872
         2      122  1873
         3      120  1874
         4      190  1875
\end{Verbatim}
            
    \begin{center}
    \adjustimage{max size={0.9\linewidth}{0.9\paperheight}}{output_33_1.png}
    \end{center}
    { \hspace*{\fill} \\}
    
    \subsection{Plot 2: Evolutie aantal actieve
ploegen}\label{plot-2-evolutie-aantal-actieve-ploegen}

    De volgende grafiek toont het aantal ploegen dat elk jaar acief is. Er
is een stijgende trend op te merken in de grafiek.

    \begin{Verbatim}[commandchars=\\\{\}]
{\color{incolor}In [{\color{incolor}45}]:} \PY{n}{column\PYZus{}names\PYZus{}vis2} \PY{o}{=}\PY{p}{[}\PY{l+s+s1}{\PYZsq{}}\PY{l+s+s1}{Teams}\PY{l+s+s1}{\PYZsq{}}\PY{p}{]}
\end{Verbatim}


    \begin{Verbatim}[commandchars=\\\{\}]
{\color{incolor}In [{\color{incolor}46}]:} \PY{c+c1}{\PYZsh{} Maak hier je eerste optimalisatie}
         \PY{k}{def} \PY{n+nf}{query\PYZus{}visualisatie2}\PY{p}{(}\PY{n}{connection}\PY{p}{,} \PY{n}{column\PYZus{}names}\PY{p}{)}\PY{p}{:}
             \PY{c+c1}{\PYZsh{} het opstellen }
             \PY{n}{query}\PY{o}{=}\PY{l+s+s2}{\PYZdq{}\PYZdq{}\PYZdq{}}
         \PY{l+s+s2}{        SELECT count(*)}
         \PY{l+s+s2}{        FROM   Teams as t, teamsfranchises as tf}
         \PY{l+s+s2}{        WHERE  tf.franchID = t.franchID AND tf.active = }\PY{l+s+s2}{\PYZsq{}}\PY{l+s+s2}{Y}\PY{l+s+s2}{\PYZsq{}}
         \PY{l+s+s2}{        GROUP BY t.yearID}
         \PY{l+s+s2}{        }\PY{l+s+s2}{\PYZdq{}\PYZdq{}\PYZdq{}}
             \PY{c+c1}{\PYZsh{} Stap 2 \PYZam{} 3}
             \PY{n}{res} \PY{o}{=} \PY{n}{run\PYZus{}query}\PY{p}{(}\PY{n}{connection}\PY{p}{,} \PY{n}{query}\PY{p}{)}         \PY{c+c1}{\PYZsh{} Query uitvoeren}
             \PY{n}{df} \PY{o}{=} \PY{n}{res\PYZus{}to\PYZus{}df}\PY{p}{(}\PY{n}{res}\PY{p}{,} \PY{n}{column\PYZus{}names}\PY{p}{)}          \PY{c+c1}{\PYZsh{} Query in DataFrame brengen}
                
             \PY{k}{return} \PY{n}{df}
\end{Verbatim}


    \begin{Verbatim}[commandchars=\\\{\}]
{\color{incolor}In [{\color{incolor}47}]:} \PY{n}{username} \PY{o}{=} \PY{l+s+s1}{\PYZsq{}}\PY{l+s+s1}{root}\PY{l+s+s1}{\PYZsq{}}      \PY{c+c1}{\PYZsh{} Vervang dit als je via een andere user queries stuurt}
         \PY{n}{hostname} \PY{o}{=} \PY{l+s+s1}{\PYZsq{}}\PY{l+s+s1}{localhost}\PY{l+s+s1}{\PYZsq{}} \PY{c+c1}{\PYZsh{} Als je een databank lokaal draait, is dit localhost.}
         \PY{n}{db} \PY{o}{=} \PY{l+s+s1}{\PYZsq{}}\PY{l+s+s1}{lahman2016test}\PY{l+s+s1}{\PYZsq{}}      \PY{c+c1}{\PYZsh{} Naam van de gegevensbank op je XAMPP Mysql server}
         
         \PY{c+c1}{\PYZsh{} We verbinden met de gegevensbank}
         \PY{n}{c} \PY{o}{=} \PY{n}{verbind\PYZus{}met\PYZus{}GB}\PY{p}{(}\PY{n}{username}\PY{p}{,} \PY{n}{hostname}\PY{p}{,} \PY{n}{db}\PY{p}{)}
\end{Verbatim}


    \begin{Verbatim}[commandchars=\\\{\}]
········

    \end{Verbatim}

    \begin{Verbatim}[commandchars=\\\{\}]
{\color{incolor}In [{\color{incolor}48}]:} \PY{c+c1}{\PYZsh{} De voorbeeldquery heeft dezelfde kolomnamen als query 1, dus we gebruiken die}
         \PY{n}{kolommen} \PY{o}{=} \PY{n}{column\PYZus{}names\PYZus{}vis2}
         
         \PY{c+c1}{\PYZsh{} Functie uitvoeren, geeft resultaat van de query in een DataFrame}
         \PY{n}{df} \PY{o}{=} \PY{n}{query\PYZus{}visualisatie2}\PY{p}{(}\PY{n}{c}\PY{p}{,} \PY{n}{column\PYZus{}names\PYZus{}vis2}\PY{p}{)}
         \PY{n}{plt}\PY{o}{.}\PY{n}{plot}\PY{p}{(}\PY{n}{df}\PY{p}{)}
         \PY{c+c1}{\PYZsh{} We inspecteren de eerste paar resultaten (voor alles te zien: laat .head() weg)}
         \PY{n}{df}\PY{o}{.}\PY{n}{head}\PY{p}{(}\PY{p}{)}
\end{Verbatim}


\begin{Verbatim}[commandchars=\\\{\}]
{\color{outcolor}Out[{\color{outcolor}48}]:}    Teams
         0      2
         1      2
         2      2
         3      2
         4      2
\end{Verbatim}
            
    \begin{center}
    \adjustimage{max size={0.9\linewidth}{0.9\paperheight}}{output_39_1.png}
    \end{center}
    { \hspace*{\fill} \\}
    

    % Add a bibliography block to the postdoc
    
    
    
    \end{document}
